
% This is the user manual for UICTHESI CLS, originally found
% at https://www.math.uic.edu/graduate/current/uicthesi, and
% modified by Pete Snyder <snyder@gmail.com> to match the
% current department requirements.

% Compile this preamble to eliminate some errors from the TeX compilation

\usepackage{mathtools}
\usepackage{amsmath}
\usepackage[makeroom]{cancel}
\usepackage{booktabs}
\usepackage{listings}
\usepackage{newlfont}
\usepackage{amsfonts}
\usepackage{amssymb}
\usepackage{ifpdf}
\usepackage{float}
\usepackage{euler}
\usepackage[xindy={glsnumbers=false},nonumberlist,acronym,nopostdot,nogroupskip,nomain]{glossaries}
\usepackage{xspace}
\usepackage[htt]{hyphenat}
\usepackage{float}
\usepackage{flushend}
\usepackage{footnote}
\usepackage{enumitem}
\usepackage{url}
\usepackage{caption}
\usepackage{graphicx}
\usepackage{keyval}
\usepackage{kvoptions}
\usepackage{fancyvrb}
\usepackage{upquote}
\usepackage{ifthen}
\usepackage{calc}
\usepackage{ifplatform}
\usepackage{pdftexcmds}
\usepackage{etoolbox}
\usepackage{xstring}
\usepackage{xcolor}
\usepackage{lineno}
\usepackage{framed}
\usepackage{shellesc}
\usepackage [english]{babel}
\usepackage [autostyle, english = american]{csquotes}
\usepackage{empheq}
\MakeOuterQuote{"}

\newglossarystyle{clong}{%
 \renewenvironment{theglossary}%
     {\begin{longtable}{p{.2\linewidth}p{.8\linewidth}}}%
     {\end{longtable}}%
  \renewcommand*{\glossaryheader}{}%
  \renewcommand*{\glsgroupheading}[1]{}%
  \renewcommand*{\glossaryentryfield}[5]{%
    \glstarget{##1}{##2} & ##3\glspostdescription\space ##5\\}%
  \renewcommand*{\glossarysubentryfield}[6]{%
     & \glstarget{##2}{\strut}##4\glspostdescription\space ##6\\}%
  %\renewcommand*{\glsgroupskip}{ & \\}%
}

\makesavenoteenv{tabular}
\makesavenoteenv{table}

% See https://www.ctan.org/pkg/glossaries for questions on this package.
% Refer to the acronyms you define here as \gls{nyc}.
%
% This make sure that this text:
%    The Ramones are from \gls{nyc}, thats right, \gls{nyc}.
% Gets output like this:
%    The Ramones are from New York City (NYC), thats right, NYC.
% And that a line in the acronyms section of your thesis has an entry like:
%    NYC         New York City
\newacronym{nyc}{NYC}{New York City}
\newacronym{rtr}{RTR}{Rocket to Russia}
\newacronym{eofc}{EOFC}{End of the Century}
\def\new@fontshape#1#2#3#4#5{\expandafter
     \edef\csname#1/#2/#3\endcsname{\expandafter\noexpand
                                 \csname #4\endcsname}}
\new@fontshape{cmr}{bx}{sc}{
      <5>cmcsc8 at 5pt%
      <6>cmcsc8 at 6pt%
      <7>cmcsc8 at 7pt%
      <8>cmcsc8%
      <9>cmcsc9%
      <10>cmcsc10%
      <11>cmcsc10 at 10.95pt%
      <12>cmcsc10 at 12pt%
      <14>cmcsc10 at 14.4pt%
      <17>cmcsc10 at 17.28%
      <20>cmcsc10 at 20.736pt%
      <25>cmcsc10 at 24.8832pt%
      }{}
\mathversion{normal}
\newcommand{\ams}{{$\cal{A}\cal{M}\cal{S}$}}
\newcommand{\amslatex}{{$\cal{A}\cal{M}\cal{S}$-\LaTeX{}}}
\newcommand{\amstex}{{$\cal{A}\cal{M}\cal{S}$-\TeX{}}}
\newcommand{\BibTeX}{{\rm B\kern-.05em{\sc i\kern-.025em b}\kern-.08em
    T\kern-.1667em\lower.7ex\hbox{E}\kern-.125emX}}
\newcommand{\uicthesi}{{$\mathbb{UICTHESI}$}}

\newcommand\bs{\char '134 }   % A backslash character for \tt font
\newcommand{\lb}{\char '173 } % A left brace character for \tt font
\newcommand{\rb}{\char '175 } % A right brace character for \tt font

% one or two other commands
\def\newfont#1#2{\@ifdefinable #1{\font #1=#2\relax}}
\def\symbol#1{\char #1\relax}

\makeglossaries

\newcommand*\widefbox[1]{\fbox{\hspace{2em}#1\hspace{2em}}}

\newacronym{PEC}{PEC}{Perfect Electrically Conducting}
\newacronym{PCF}{PCF}{Parabolic Cylindrical Wave Function}